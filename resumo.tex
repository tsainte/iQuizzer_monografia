\chapter*{Resumo}
\label{chp:resumo}%%
\thispagestyle{empty}

\setcounter{footnote}{0}

Com o crescente desenvolvimento da mobilidade e da ubiqüidade, um grande número de aplicações Web têm sido desenvolvidas, sendo as plataformas móveis \emph{iOS} e \emph{Android} algumas das tecnologias de destaque à época em que este trabalho foi escrito. Paralelamente, muitos recursos distribuídos se conjugam através de técnicas e organizações específicas, como as arquiteturas orientadas a serviços. Para exemplificar, algumas redes sociais populares, como \emph{Twitter} e \emph{Foursquare}, possuem interfaces de comunicação baseadas no uso de \emph{REST} para exportar funcionalidades, em forma de serviços para outras aplicações. 

Este trabalho tem por objetivo apresentar um estudo em que se mostra o desenvolvimento de uma aplicação Web que integra aplicações \emph{iOS} e \emph{Android} a um aplicativo web \emph{RESTful}. Para realizar o estudo, é criada uma pequena rede social de criação de quizzes, utilizando \emph{Ruby on Rails} e um serviço de PaaS, como o \emph{Heroku}.


\noindent \textbf{Palavras-chave: iOS, Android, Ruby on Rails, REST, Computação em Nuvem}