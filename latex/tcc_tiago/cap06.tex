\chapter{Conclusão} \label{CHP:CONC}


\section{Contribuições}
	Este trabalho trouxe duas contribuições importantes: a primeira, referente ao aplicativo iQuizzer; a segunda, referente a documentação de configuração e implementação de arquitetura de aplicativo móvel utilizando web services.
	
	O aplicativo \emph{iQuizzer}, disponibilizado nas versões web e mobile, estimula a busca por conhecimento de um determinado assunto, tanto para quem está criando o quiz quanto para quem está jogando. Além disso, permite que enquetes sejam feitas de maneira rápida, com resultados sendo avaliados em uma plataforma web.
	
	A documentação gerada fornece uma base para muitas das aplicações que estão surgindo utilizando a abordagem de web service com computação móvel. Tal abordagem vem sendo utilizadas por um grande número de \emph{Startups}, as quais necessitam, essencialmente, desenvolver e avaliar seus produtos de maneira rápida, mesmo que o produto ainda esteja em fase de desenvolvimento.

\section{Limitações}
	A aplicação \emph{iQuizzer} tinha como um dos objetivos ser a validação de uma idéia onde quizzes são criados e acompanhados na web e jogados via mobile. Como o intuito era a validação, algumas funcionalidades não foram implementadas por não interferirem diretamente na validação da idéia. 
	
	Algumas das funcionalidades não implementadas são limitações do projeto, como: a não atualização dos quizzes depois de baixados, mesmo que modificados na web; a não adoção de conexões assíncronas; a falta de integração com as redes sociais; a falta de criptografia na troca de mensagens.
	
	No que concerne as plataformas de desenvolvimento mobile, existem duas limitações. O aplicativo desenvolvido em \emph{iOS} não funciona em versões inferiores ao \emph{iOS 5.0}, devido ao uso de bibliotecas exclusivas dessa versão. De maneira similar, o aplicativo desenvolvido em \emph{Android} não funciona em versões inferiores ao \emph{Android 2.3 Gingerbread}.
	
 
\section{Trabalhos futuros}
	A fim de lançar a plataforma \emph{iQuizzer} no mercado, de maneira escalável e competitiva, algumas funcionalidades deveriam ser implementadas:
\begin{itemize}
\item Integração com as redes sociais: o módulo web poderia ser disponível como um aplicativo para \emph{Facebook}, onde o controle de usuários seria feito pela própria conta do usuário na rede. Além disso, as ações do usuário, tanto na parte web como na parte mobile, podem corresponder a ações no Facebook, de modo a divulgar a plataforma de forma viralizada.
\item Segurança: toda a troca de mensagens pode ser feita utilizando protocolos seguros, como \ac{SSH}. Dessa forma, poderia garantir a inviabilidade da informação trocada.
\item Elementos de rede social: ações como ``curtir'', ``comentar'' e ``seguir usuário'' podem ser colocadas dentro da plataforma, de modo a aumentar a experiência do usuário em relação a outros usuários.
\item Disponibilização do jogo para outras plataformas: a função de jogar um quiz pode ser disponibilizada na web, no \emph{Windows Phone}, entre outros.
\item Conexões assíncronas, de modo a otimizar a experiência do usuário em relação a usabilidade.
\item Melhorias no layout: no estado atual, a aplicação não possui uma identidade visual, o que é de vital importância para o sucesso comercial do aplicativo.
\item Notificações e controles de atualização de quizzes.
\item Monetização, com propagandas ou venda de quizzes: pode-se monetizar a aplicação colocando banners, tanto na web como no mobile. Poderia, também, ser criado um mini-sistema de venda de quizzes, onde os criadores de cada quiz lucrassem com a compra de quizzes pelos jogadores.
\end{itemize}
